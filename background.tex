\documentclass[main.tex]{subfiles}
\begin{comment}
\MakeOuterQuote{"}
\usepackage{multicol}
\setlength{\columnsep}{1cm}
\end{comment}

\begin{document}
%\normalsize
%\begin{multicols}{2}[]
\begin{center}\section{Background}\end{center}
\paragraph{}
Network security is one of the most crucial and frequently neglected aspect in a computer network. A network security breach could impact all the actors in that network, from consumers, companies, and as well as governments. For consumers, networks could include their online identities leaving them victims of attackers who are interested in stealing information such as, credit card numbers, passwords, and other data that can cause a tremendous amount of damage if it was in the wrong hands. For companies and corporations, the damage could be much worse from losing sensitive data to another competitor. Furthermore, a weak network security could influence governments, as attackers could gain access to confidential military documents as well as to the financial records which could be used against that government itself.\\
Network attacks are expected and are more common than you think. And it is obvious now that a new course of action must be taken to prevent them.
\paragraph{}
Most of the cybersecurity research that has been, or is being done direct its attention on preventing attackers from breaching into the network. And the most commonly practiced method for that is the Intrusion Detection System (IDS) \cite{liu2010data}. The goal of IDS is to give attackers several security layers to beat before breaching the network. Of course, this defensive tool is quite necessary but it is even more essential to look at other lines of defence as well. Since these protective methods cannot examine all network traffic without costly hardware or high overhead that will result in network slowdowns. Plus, in reality, these defensive tools can never absolutely protect the network as the most eager and enthusiastic hackers will not stop trying at the sign of a powerful defence especially if they are trying to breach a military or governmental network that will give them a great advantage in the times of war and political activities. Hence, eventually, they will find a way around these tools to breach the network. Based on the above, it is not only important to look into the incoming traffic for abnormalities that could be an evidence of network breach but also to examine the outgoing traffic for the effect of a network breach. Promptly, in order to earn from a network breach, a hacker must be able to find a way to transfer data from the attacked machine to his server and this is known as Data Exfiltration. Some tools do exist to block sensitive data from leaving the network. These tools are referred to as Data Leak Prevention $($DLP$)$ tools. However, these tools are targeted and are an expression or keyword$-$based algorithms, which an intruder can easily overcome.\
\end{document}